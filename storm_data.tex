\documentclass[]{article}
\usepackage{lmodern}
\usepackage{amssymb,amsmath}
\usepackage{ifxetex,ifluatex}
\usepackage{fixltx2e} % provides \textsubscript
\ifnum 0\ifxetex 1\fi\ifluatex 1\fi=0 % if pdftex
  \usepackage[T1]{fontenc}
  \usepackage[utf8]{inputenc}
\else % if luatex or xelatex
  \ifxetex
    \usepackage{mathspec}
  \else
    \usepackage{fontspec}
  \fi
  \defaultfontfeatures{Ligatures=TeX,Scale=MatchLowercase}
\fi
% use upquote if available, for straight quotes in verbatim environments
\IfFileExists{upquote.sty}{\usepackage{upquote}}{}
% use microtype if available
\IfFileExists{microtype.sty}{%
\usepackage{microtype}
\UseMicrotypeSet[protrusion]{basicmath} % disable protrusion for tt fonts
}{}
\usepackage[margin=1in]{geometry}
\usepackage{hyperref}
\hypersetup{unicode=true,
            pdfborder={0 0 0},
            breaklinks=true}
\urlstyle{same}  % don't use monospace font for urls
\usepackage{color}
\usepackage{fancyvrb}
\newcommand{\VerbBar}{|}
\newcommand{\VERB}{\Verb[commandchars=\\\{\}]}
\DefineVerbatimEnvironment{Highlighting}{Verbatim}{commandchars=\\\{\}}
% Add ',fontsize=\small' for more characters per line
\usepackage{framed}
\definecolor{shadecolor}{RGB}{248,248,248}
\newenvironment{Shaded}{\begin{snugshade}}{\end{snugshade}}
\newcommand{\KeywordTok}[1]{\textcolor[rgb]{0.13,0.29,0.53}{\textbf{#1}}}
\newcommand{\DataTypeTok}[1]{\textcolor[rgb]{0.13,0.29,0.53}{#1}}
\newcommand{\DecValTok}[1]{\textcolor[rgb]{0.00,0.00,0.81}{#1}}
\newcommand{\BaseNTok}[1]{\textcolor[rgb]{0.00,0.00,0.81}{#1}}
\newcommand{\FloatTok}[1]{\textcolor[rgb]{0.00,0.00,0.81}{#1}}
\newcommand{\ConstantTok}[1]{\textcolor[rgb]{0.00,0.00,0.00}{#1}}
\newcommand{\CharTok}[1]{\textcolor[rgb]{0.31,0.60,0.02}{#1}}
\newcommand{\SpecialCharTok}[1]{\textcolor[rgb]{0.00,0.00,0.00}{#1}}
\newcommand{\StringTok}[1]{\textcolor[rgb]{0.31,0.60,0.02}{#1}}
\newcommand{\VerbatimStringTok}[1]{\textcolor[rgb]{0.31,0.60,0.02}{#1}}
\newcommand{\SpecialStringTok}[1]{\textcolor[rgb]{0.31,0.60,0.02}{#1}}
\newcommand{\ImportTok}[1]{#1}
\newcommand{\CommentTok}[1]{\textcolor[rgb]{0.56,0.35,0.01}{\textit{#1}}}
\newcommand{\DocumentationTok}[1]{\textcolor[rgb]{0.56,0.35,0.01}{\textbf{\textit{#1}}}}
\newcommand{\AnnotationTok}[1]{\textcolor[rgb]{0.56,0.35,0.01}{\textbf{\textit{#1}}}}
\newcommand{\CommentVarTok}[1]{\textcolor[rgb]{0.56,0.35,0.01}{\textbf{\textit{#1}}}}
\newcommand{\OtherTok}[1]{\textcolor[rgb]{0.56,0.35,0.01}{#1}}
\newcommand{\FunctionTok}[1]{\textcolor[rgb]{0.00,0.00,0.00}{#1}}
\newcommand{\VariableTok}[1]{\textcolor[rgb]{0.00,0.00,0.00}{#1}}
\newcommand{\ControlFlowTok}[1]{\textcolor[rgb]{0.13,0.29,0.53}{\textbf{#1}}}
\newcommand{\OperatorTok}[1]{\textcolor[rgb]{0.81,0.36,0.00}{\textbf{#1}}}
\newcommand{\BuiltInTok}[1]{#1}
\newcommand{\ExtensionTok}[1]{#1}
\newcommand{\PreprocessorTok}[1]{\textcolor[rgb]{0.56,0.35,0.01}{\textit{#1}}}
\newcommand{\AttributeTok}[1]{\textcolor[rgb]{0.77,0.63,0.00}{#1}}
\newcommand{\RegionMarkerTok}[1]{#1}
\newcommand{\InformationTok}[1]{\textcolor[rgb]{0.56,0.35,0.01}{\textbf{\textit{#1}}}}
\newcommand{\WarningTok}[1]{\textcolor[rgb]{0.56,0.35,0.01}{\textbf{\textit{#1}}}}
\newcommand{\AlertTok}[1]{\textcolor[rgb]{0.94,0.16,0.16}{#1}}
\newcommand{\ErrorTok}[1]{\textcolor[rgb]{0.64,0.00,0.00}{\textbf{#1}}}
\newcommand{\NormalTok}[1]{#1}
\usepackage{graphicx,grffile}
\makeatletter
\def\maxwidth{\ifdim\Gin@nat@width>\linewidth\linewidth\else\Gin@nat@width\fi}
\def\maxheight{\ifdim\Gin@nat@height>\textheight\textheight\else\Gin@nat@height\fi}
\makeatother
% Scale images if necessary, so that they will not overflow the page
% margins by default, and it is still possible to overwrite the defaults
% using explicit options in \includegraphics[width, height, ...]{}
\setkeys{Gin}{width=\maxwidth,height=\maxheight,keepaspectratio}
\IfFileExists{parskip.sty}{%
\usepackage{parskip}
}{% else
\setlength{\parindent}{0pt}
\setlength{\parskip}{6pt plus 2pt minus 1pt}
}
\setlength{\emergencystretch}{3em}  % prevent overfull lines
\providecommand{\tightlist}{%
  \setlength{\itemsep}{0pt}\setlength{\parskip}{0pt}}
\setcounter{secnumdepth}{0}
% Redefines (sub)paragraphs to behave more like sections
\ifx\paragraph\undefined\else
\let\oldparagraph\paragraph
\renewcommand{\paragraph}[1]{\oldparagraph{#1}\mbox{}}
\fi
\ifx\subparagraph\undefined\else
\let\oldsubparagraph\subparagraph
\renewcommand{\subparagraph}[1]{\oldsubparagraph{#1}\mbox{}}
\fi

%%% Use protect on footnotes to avoid problems with footnotes in titles
\let\rmarkdownfootnote\footnote%
\def\footnote{\protect\rmarkdownfootnote}

%%% Change title format to be more compact
\usepackage{titling}

% Create subtitle command for use in maketitle
\providecommand{\subtitle}[1]{
  \posttitle{
    \begin{center}\large#1\end{center}
    }
}

\setlength{\droptitle}{-2em}

  \title{}
    \pretitle{\vspace{\droptitle}}
  \posttitle{}
    \author{}
    \preauthor{}\postauthor{}
    \date{}
    \predate{}\postdate{}
  

\begin{document}

\subsection{Reproducible Research: Course Project
2}\label{reproducible-research-course-project-2}

\subsection{Author: Shayan (Sean)
Taheri}\label{author-shayan-sean-taheri}

\subsection{Impact of Severe Weather Events on Public Health and Economy
in the United
States}\label{impact-of-severe-weather-events-on-public-health-and-economy-in-the-united-states}

\subsubsection{Synonpsis}\label{synonpsis}

Our goal in this project is analysis of the impact of different weather
events on public health and economy based on the storm database cllected
from U.S. National Occeanic and Atmospheric Administration's (NOAA) from
1950 - 2011.

The estimates of fatalities, injuries, property, and crop damage are
used in cropping damage to harmful to the population healthy and
economy. From these data, we found that excessive heat and tornado are
most harmful with respect to population health, while flood, drought,
and hurricane/typhoon have the greatest economic consequences.

\subsubsection{Basic settings}\label{basic-settings}

\begin{Shaded}
\begin{Highlighting}[]
\NormalTok{echo =}\StringTok{ }\OtherTok{TRUE}  \CommentTok{# Always make code visible}
\KeywordTok{options}\NormalTok{(}\DataTypeTok{scipen =} \DecValTok{1}\NormalTok{)  }\CommentTok{# Turn off scientific notations for numbers}
\KeywordTok{library}\NormalTok{(R.utils)}
\end{Highlighting}
\end{Shaded}

\begin{verbatim}
## Warning: package 'R.utils' was built under R version 3.5.3
\end{verbatim}

\begin{verbatim}
## Loading required package: R.oo
\end{verbatim}

\begin{verbatim}
## Loading required package: R.methodsS3
\end{verbatim}

\begin{verbatim}
## R.methodsS3 v1.7.1 (2016-02-15) successfully loaded. See ?R.methodsS3 for help.
\end{verbatim}

\begin{verbatim}
## R.oo v1.22.0 (2018-04-21) successfully loaded. See ?R.oo for help.
\end{verbatim}

\begin{verbatim}
## 
## Attaching package: 'R.oo'
\end{verbatim}

\begin{verbatim}
## The following objects are masked from 'package:methods':
## 
##     getClasses, getMethods
\end{verbatim}

\begin{verbatim}
## The following objects are masked from 'package:base':
## 
##     attach, detach, gc, load, save
\end{verbatim}

\begin{verbatim}
## R.utils v2.8.0 successfully loaded. See ?R.utils for help.
\end{verbatim}

\begin{verbatim}
## 
## Attaching package: 'R.utils'
\end{verbatim}

\begin{verbatim}
## The following object is masked from 'package:utils':
## 
##     timestamp
\end{verbatim}

\begin{verbatim}
## The following objects are masked from 'package:base':
## 
##     cat, commandArgs, getOption, inherits, isOpen, parse, warnings
\end{verbatim}

\begin{Shaded}
\begin{Highlighting}[]
\KeywordTok{library}\NormalTok{(ggplot2)}
\end{Highlighting}
\end{Shaded}

\begin{verbatim}
## Warning: package 'ggplot2' was built under R version 3.5.3
\end{verbatim}

\begin{Shaded}
\begin{Highlighting}[]
\KeywordTok{library}\NormalTok{(plyr)}
\KeywordTok{require}\NormalTok{(gridExtra)}
\end{Highlighting}
\end{Shaded}

\begin{verbatim}
## Loading required package: gridExtra
\end{verbatim}

\subsubsection{Data Processing}\label{data-processing}

Let's download the data file and uncompress it in the first step.

\begin{Shaded}
\begin{Highlighting}[]
\KeywordTok{getwd}\NormalTok{()}
\end{Highlighting}
\end{Shaded}

\begin{verbatim}
## [1] "C:/Users/shaya/Desktop/Reprocible_Research_Course_Project_2"
\end{verbatim}

\begin{Shaded}
\begin{Highlighting}[]
\KeywordTok{setwd}\NormalTok{(}\StringTok{"C:/Users/shaya/Desktop/Reprocible_Research_Course_Project_2"}\NormalTok{)}

\ControlFlowTok{if}\NormalTok{ (}\OperatorTok{!}\StringTok{"stormData.csv.bz2"} \OperatorTok\StringTok{ }\KeywordTok{dir}\NormalTok{(}\StringTok{"./data/"}\NormalTok{)) \{}
    \KeywordTok{print}\NormalTok{(}\StringTok{"hhhh"}\NormalTok{)}
    \KeywordTok{download.file}\NormalTok{(}\StringTok{"http://d396qusza40orc.cloudfront.net/repdata%2Fdata%2FStormData.csv.bz2"}\NormalTok{, }\DataTypeTok{destfile =} \StringTok{"data/stormData.csv.bz2"}\NormalTok{)}
    \KeywordTok{bunzip2}\NormalTok{(}\StringTok{"data/stormData.csv.bz2"}\NormalTok{, }\DataTypeTok{overwrite=}\NormalTok{T, }\DataTypeTok{remove=}\NormalTok{F)}
\NormalTok{\}}
\end{Highlighting}
\end{Shaded}

Then, we read the generated csv file. If the data already exists in the
working environment, we do not need to load it again. Otherwise, we read
the csv file.

Next, the generated CSV file is read if it does not exist in the working
environment.

\begin{Shaded}
\begin{Highlighting}[]
\ControlFlowTok{if}\NormalTok{ (}\OperatorTok{!}\StringTok{"stormData"} \OperatorTok\StringTok{ }\KeywordTok{ls}\NormalTok{()) \{}
\NormalTok{    stormData <-}\StringTok{ }\KeywordTok{read.csv}\NormalTok{(}\StringTok{"data/stormData.csv"}\NormalTok{, }\DataTypeTok{sep =} \StringTok{","}\NormalTok{)}
\NormalTok{\}}
\KeywordTok{dim}\NormalTok{(stormData)}
\end{Highlighting}
\end{Shaded}

\begin{verbatim}
## [1] 902297     37
\end{verbatim}

\begin{Shaded}
\begin{Highlighting}[]
\KeywordTok{head}\NormalTok{(stormData, }\DataTypeTok{n =} \DecValTok{2}\NormalTok{)}
\end{Highlighting}
\end{Shaded}

\begin{verbatim}
##   STATE__          BGN_DATE BGN_TIME TIME_ZONE COUNTY COUNTYNAME STATE
## 1       1 4/18/1950 0:00:00     0130       CST     97     MOBILE    AL
## 2       1 4/18/1950 0:00:00     0145       CST      3    BALDWIN    AL
##    EVTYPE BGN_RANGE BGN_AZI BGN_LOCATI END_DATE END_TIME COUNTY_END
## 1 TORNADO         0                                               0
## 2 TORNADO         0                                               0
##   COUNTYENDN END_RANGE END_AZI END_LOCATI LENGTH WIDTH F MAG FATALITIES
## 1         NA         0                        14   100 3   0          0
## 2         NA         0                         2   150 2   0          0
##   INJURIES PROPDMG PROPDMGEXP CROPDMG CROPDMGEXP WFO STATEOFFIC ZONENAMES
## 1       15    25.0          K       0                                    
## 2        0     2.5          K       0                                    
##   LATITUDE LONGITUDE LATITUDE_E LONGITUDE_ REMARKS REFNUM
## 1     3040      8812       3051       8806              1
## 2     3042      8755          0          0              2
\end{verbatim}

There are 902297 rows and 37 columns in total. The database start is in
the year of 1950 and ends in November 2011. In the earlier years of the
database there are generally fewer events recorded, most likely due to a
lack of good records. More recent years should be considered more
complete.

\begin{Shaded}
\begin{Highlighting}[]
\ControlFlowTok{if}\NormalTok{ (}\KeywordTok{dim}\NormalTok{(stormData)[}\DecValTok{2}\NormalTok{] }\OperatorTok{==}\StringTok{ }\DecValTok{37}\NormalTok{) \{}
\NormalTok{    stormData}\OperatorTok{$}\NormalTok{year <-}\StringTok{ }\KeywordTok{as.numeric}\NormalTok{(}\KeywordTok{format}\NormalTok{(}\KeywordTok{as.Date}\NormalTok{(stormData}\OperatorTok{$}\NormalTok{BGN_DATE, }\DataTypeTok{format =} \StringTok{"%m/%d/%Y %H:%M:%S"}\NormalTok{), }\StringTok{"%Y"}\NormalTok{))}
\NormalTok{\}}
\KeywordTok{hist}\NormalTok{(stormData}\OperatorTok{$}\NormalTok{year, }\DataTypeTok{breaks =} \DecValTok{30}\NormalTok{)}
\end{Highlighting}
\end{Shaded}

\includegraphics{storm_data_files/figure-latex/unnamed-chunk-4-1.pdf}

Based on the delievered histogram, it can be seen that the number of
events tracked starts to increase significantly specifically around
1995. So, we use the subset of the data from 1990 to 2011. to get most
out of good records.

\begin{Shaded}
\begin{Highlighting}[]
\NormalTok{storm <-}\StringTok{ }\NormalTok{stormData[stormData}\OperatorTok{$}\NormalTok{year }\OperatorTok{>=}\StringTok{ }\DecValTok{1995}\NormalTok{, ]}
\KeywordTok{dim}\NormalTok{(storm)}
\end{Highlighting}
\end{Shaded}

\begin{verbatim}
## [1] 681500     38
\end{verbatim}

Now, there are 681500 rows and 38 columns in total.

\paragraph{Impact on Public Health}\label{impact-on-public-health}

The number of \textbf{fatalities} and \textbf{injuries} are reviewd.They
are caused by the sever weather events. We would like to get the first
15 most severe types of weather events.

\begin{Shaded}
\begin{Highlighting}[]
\NormalTok{sortHelper <-}\StringTok{ }\ControlFlowTok{function}\NormalTok{(fieldName, }\DataTypeTok{top =} \DecValTok{15}\NormalTok{, }\DataTypeTok{dataset =}\NormalTok{ stormData) \{}
\NormalTok{    index <-}\StringTok{ }\KeywordTok{which}\NormalTok{(}\KeywordTok{colnames}\NormalTok{(dataset) }\OperatorTok{==}\StringTok{ }\NormalTok{fieldName)}
\NormalTok{    field <-}\StringTok{ }\KeywordTok{aggregate}\NormalTok{(dataset[, index], }\DataTypeTok{by =} \KeywordTok{list}\NormalTok{(dataset}\OperatorTok{$}\NormalTok{EVTYPE), }\DataTypeTok{FUN =} \StringTok{"sum"}\NormalTok{)}
    \KeywordTok{names}\NormalTok{(field) <-}\StringTok{ }\KeywordTok{c}\NormalTok{(}\StringTok{"EVTYPE"}\NormalTok{, fieldName)}
\NormalTok{    field <-}\StringTok{ }\KeywordTok{arrange}\NormalTok{(field, field[, }\DecValTok{2}\NormalTok{], }\DataTypeTok{decreasing =}\NormalTok{ T)}
\NormalTok{    field <-}\StringTok{ }\KeywordTok{head}\NormalTok{(field, }\DataTypeTok{n =}\NormalTok{ top)}
\NormalTok{    field <-}\StringTok{ }\KeywordTok{within}\NormalTok{(field, EVTYPE <-}\StringTok{ }\KeywordTok{factor}\NormalTok{(}\DataTypeTok{x =}\NormalTok{ EVTYPE, }\DataTypeTok{levels =}\NormalTok{ field}\OperatorTok{$}\NormalTok{EVTYPE))}
    \KeywordTok{return}\NormalTok{(field)}
\NormalTok{\}}

\NormalTok{fatalities <-}\StringTok{ }\KeywordTok{sortHelper}\NormalTok{(}\StringTok{"FATALITIES"}\NormalTok{, }\DataTypeTok{dataset =}\NormalTok{ storm)}
\NormalTok{injuries <-}\StringTok{ }\KeywordTok{sortHelper}\NormalTok{(}\StringTok{"INJURIES"}\NormalTok{, }\DataTypeTok{dataset =}\NormalTok{ storm)}
\end{Highlighting}
\end{Shaded}

\paragraph{Impact on Economy}\label{impact-on-economy}

The \textbf{property damage} and \textbf{crop damage} data are converted
into comparable numerical forms. This conversion is in accordance with
the meaning of units described in the code book
(\href{http://ire.org/nicar/database-library/databases/storm-events/}{Storm
Events}).

Both \texttt{PROPDMGEXP} and \texttt{CROPDMGEXP} columns record a
multiplier for each observation where we have Hundred (H), Thousand (K),
Million (M) and Billion (B).

\begin{Shaded}
\begin{Highlighting}[]
\NormalTok{convertHelper <-}\StringTok{ }\ControlFlowTok{function}\NormalTok{(}\DataTypeTok{dataset =}\NormalTok{ storm, fieldName, newFieldName) \{}
\NormalTok{    totalLen <-}\StringTok{ }\KeywordTok{dim}\NormalTok{(dataset)[}\DecValTok{2}\NormalTok{]}
\NormalTok{    index <-}\StringTok{ }\KeywordTok{which}\NormalTok{(}\KeywordTok{colnames}\NormalTok{(dataset) }\OperatorTok{==}\StringTok{ }\NormalTok{fieldName)}
\NormalTok{    dataset[, index] <-}\StringTok{ }\KeywordTok{as.character}\NormalTok{(dataset[, index])}
\NormalTok{    logic <-}\StringTok{ }\OperatorTok{!}\KeywordTok{is.na}\NormalTok{(}\KeywordTok{toupper}\NormalTok{(dataset[, index]))}
\NormalTok{    dataset[logic }\OperatorTok{&}\StringTok{ }\KeywordTok{toupper}\NormalTok{(dataset[, index]) }\OperatorTok{==}\StringTok{ "B"}\NormalTok{, index] <-}\StringTok{ "9"}
\NormalTok{    dataset[logic }\OperatorTok{&}\StringTok{ }\KeywordTok{toupper}\NormalTok{(dataset[, index]) }\OperatorTok{==}\StringTok{ "M"}\NormalTok{, index] <-}\StringTok{ "6"}
\NormalTok{    dataset[logic }\OperatorTok{&}\StringTok{ }\KeywordTok{toupper}\NormalTok{(dataset[, index]) }\OperatorTok{==}\StringTok{ "K"}\NormalTok{, index] <-}\StringTok{ "3"}
\NormalTok{    dataset[logic }\OperatorTok{&}\StringTok{ }\KeywordTok{toupper}\NormalTok{(dataset[, index]) }\OperatorTok{==}\StringTok{ "H"}\NormalTok{, index] <-}\StringTok{ "2"}
\NormalTok{    dataset[logic }\OperatorTok{&}\StringTok{ }\KeywordTok{toupper}\NormalTok{(dataset[, index]) }\OperatorTok{==}\StringTok{ ""}\NormalTok{, index] <-}\StringTok{ "0"}
\NormalTok{    dataset[, index] <-}\StringTok{ }\KeywordTok{as.numeric}\NormalTok{(dataset[, index])}
\NormalTok{    dataset[}\KeywordTok{is.na}\NormalTok{(dataset[, index]), index] <-}\StringTok{ }\DecValTok{0}
\NormalTok{    dataset <-}\StringTok{ }\KeywordTok{cbind}\NormalTok{(dataset, dataset[, index }\OperatorTok{-}\StringTok{ }\DecValTok{1}\NormalTok{] }\OperatorTok{*}\StringTok{ }\DecValTok{10}\OperatorTok{^}\NormalTok{dataset[, index])}
    \KeywordTok{names}\NormalTok{(dataset)[totalLen }\OperatorTok{+}\StringTok{ }\DecValTok{1}\NormalTok{] <-}\StringTok{ }\NormalTok{newFieldName}
    \KeywordTok{return}\NormalTok{(dataset)}
\NormalTok{\}}

\NormalTok{storm <-}\StringTok{ }\KeywordTok{convertHelper}\NormalTok{(storm, }\StringTok{"PROPDMGEXP"}\NormalTok{, }\StringTok{"propertyDamage"}\NormalTok{)}
\end{Highlighting}
\end{Shaded}

\begin{verbatim}
## Warning in convertHelper(storm, "PROPDMGEXP", "propertyDamage"): NAs
## introduced by coercion
\end{verbatim}

\begin{Shaded}
\begin{Highlighting}[]
\NormalTok{storm <-}\StringTok{ }\KeywordTok{convertHelper}\NormalTok{(storm, }\StringTok{"CROPDMGEXP"}\NormalTok{, }\StringTok{"cropDamage"}\NormalTok{)}
\end{Highlighting}
\end{Shaded}

\begin{verbatim}
## Warning in convertHelper(storm, "CROPDMGEXP", "cropDamage"): NAs introduced
## by coercion
\end{verbatim}

\begin{Shaded}
\begin{Highlighting}[]
\KeywordTok{names}\NormalTok{(storm)}
\end{Highlighting}
\end{Shaded}

\begin{verbatim}
##  [1] "STATE__"        "BGN_DATE"       "BGN_TIME"       "TIME_ZONE"     
##  [5] "COUNTY"         "COUNTYNAME"     "STATE"          "EVTYPE"        
##  [9] "BGN_RANGE"      "BGN_AZI"        "BGN_LOCATI"     "END_DATE"      
## [13] "END_TIME"       "COUNTY_END"     "COUNTYENDN"     "END_RANGE"     
## [17] "END_AZI"        "END_LOCATI"     "LENGTH"         "WIDTH"         
## [21] "F"              "MAG"            "FATALITIES"     "INJURIES"      
## [25] "PROPDMG"        "PROPDMGEXP"     "CROPDMG"        "CROPDMGEXP"    
## [29] "WFO"            "STATEOFFIC"     "ZONENAMES"      "LATITUDE"      
## [33] "LONGITUDE"      "LATITUDE_E"     "LONGITUDE_"     "REMARKS"       
## [37] "REFNUM"         "year"           "propertyDamage" "cropDamage"
\end{verbatim}

\begin{Shaded}
\begin{Highlighting}[]
\KeywordTok{options}\NormalTok{(}\DataTypeTok{scipen=}\DecValTok{999}\NormalTok{)}
\NormalTok{property <-}\StringTok{ }\KeywordTok{sortHelper}\NormalTok{(}\StringTok{"propertyDamage"}\NormalTok{, }\DataTypeTok{dataset =}\NormalTok{ storm)}
\NormalTok{crop <-}\StringTok{ }\KeywordTok{sortHelper}\NormalTok{(}\StringTok{"cropDamage"}\NormalTok{, }\DataTypeTok{dataset =}\NormalTok{ storm)}
\end{Highlighting}
\end{Shaded}

\subsubsection{Results}\label{results}

As for the impact on public health, we have got two sorted lists of
severe weather events below by the number of people badly affected.

\begin{Shaded}
\begin{Highlighting}[]
\NormalTok{fatalities}
\end{Highlighting}
\end{Shaded}

\begin{verbatim}
##               EVTYPE FATALITIES
## 1     EXCESSIVE HEAT       1903
## 2            TORNADO       1545
## 3        FLASH FLOOD        934
## 4               HEAT        924
## 5          LIGHTNING        729
## 6              FLOOD        423
## 7        RIP CURRENT        360
## 8          HIGH WIND        241
## 9          TSTM WIND        241
## 10         AVALANCHE        223
## 11      RIP CURRENTS        204
## 12      WINTER STORM        195
## 13         HEAT WAVE        161
## 14 THUNDERSTORM WIND        131
## 15      EXTREME COLD        126
\end{verbatim}

\begin{Shaded}
\begin{Highlighting}[]
\NormalTok{injuries}
\end{Highlighting}
\end{Shaded}

\begin{verbatim}
##               EVTYPE INJURIES
## 1            TORNADO    21765
## 2              FLOOD     6769
## 3     EXCESSIVE HEAT     6525
## 4          LIGHTNING     4631
## 5          TSTM WIND     3630
## 6               HEAT     2030
## 7        FLASH FLOOD     1734
## 8  THUNDERSTORM WIND     1426
## 9       WINTER STORM     1298
## 10 HURRICANE/TYPHOON     1275
## 11         HIGH WIND     1093
## 12              HAIL      916
## 13          WILDFIRE      911
## 14        HEAVY SNOW      751
## 15               FOG      718
\end{verbatim}

A pair of graph of total fatalities nand total injuries affected by
these severe weather events is shown in the following.

\begin{Shaded}
\begin{Highlighting}[]
\NormalTok{fatalitiesPlot <-}\StringTok{ }\KeywordTok{qplot}\NormalTok{(EVTYPE, }\DataTypeTok{data =}\NormalTok{ fatalities, }\DataTypeTok{weight =}\NormalTok{ FATALITIES, }\DataTypeTok{geom =} \StringTok{"bar"}\NormalTok{, }\DataTypeTok{binwidth =} \OtherTok{NULL}\NormalTok{) }\OperatorTok{+}
\StringTok{    }\KeywordTok{scale_y_continuous}\NormalTok{(}\StringTok{"Number of Fatalities"}\NormalTok{) }\OperatorTok{+}\StringTok{ }
\StringTok{    }\KeywordTok{theme}\NormalTok{(}\DataTypeTok{axis.text.x =} \KeywordTok{element_text}\NormalTok{(}\DataTypeTok{angle =} \DecValTok{45}\NormalTok{, }
    \DataTypeTok{hjust =} \DecValTok{1}\NormalTok{)) }\OperatorTok{+}\StringTok{ }\KeywordTok{xlab}\NormalTok{(}\StringTok{"Severe Weather Type"}\NormalTok{) }\OperatorTok{+}\StringTok{ }
\StringTok{    }\KeywordTok{ggtitle}\NormalTok{(}\StringTok{"Total Fatalities by Severe Weather}\CharTok{\textbackslash{}n}\StringTok{ Events in the U.S.}\CharTok{\textbackslash{}n}\StringTok{ from 1995 - 2011"}\NormalTok{)}
\NormalTok{injuriesPlot <-}\StringTok{ }\KeywordTok{qplot}\NormalTok{(EVTYPE, }\DataTypeTok{data =}\NormalTok{ injuries, }\DataTypeTok{weight =}\NormalTok{ INJURIES, }\DataTypeTok{geom =} \StringTok{"bar"}\NormalTok{, }\DataTypeTok{binwidth =} \OtherTok{NULL}\NormalTok{) }\OperatorTok{+}\StringTok{ }
\StringTok{    }\KeywordTok{scale_y_continuous}\NormalTok{(}\StringTok{"Number of Injuries"}\NormalTok{) }\OperatorTok{+}\StringTok{ }
\StringTok{    }\KeywordTok{theme}\NormalTok{(}\DataTypeTok{axis.text.x =} \KeywordTok{element_text}\NormalTok{(}\DataTypeTok{angle =} \DecValTok{45}\NormalTok{, }
    \DataTypeTok{hjust =} \DecValTok{1}\NormalTok{)) }\OperatorTok{+}\StringTok{ }\KeywordTok{xlab}\NormalTok{(}\StringTok{"Severe Weather Type"}\NormalTok{) }\OperatorTok{+}\StringTok{ }
\StringTok{    }\KeywordTok{ggtitle}\NormalTok{(}\StringTok{"Total Injuries by Severe Weather}\CharTok{\textbackslash{}n}\StringTok{ Events in the U.S.}\CharTok{\textbackslash{}n}\StringTok{ from 1995 - 2011"}\NormalTok{)}
\KeywordTok{grid.arrange}\NormalTok{(fatalitiesPlot, injuriesPlot, }\DataTypeTok{ncol =} \DecValTok{2}\NormalTok{)}
\end{Highlighting}
\end{Shaded}

\includegraphics{storm_data_files/figure-latex/unnamed-chunk-9-1.pdf}

The histogram represented in above demonstrates the \textbf{excessive
heat} and \textbf{tornado} cause most fatalities; \textbf{excessive
heat} and \textbf{tornado} cause most fatalities; \textbf{tornato}
causes most injuries in the United States from 1995 to 2011.

Two sorted lists by the amount of money cost by damages are provided in
order to show the impact on economy.

\begin{Shaded}
\begin{Highlighting}[]
\NormalTok{property}
\end{Highlighting}
\end{Shaded}

\begin{verbatim}
##               EVTYPE propertyDamage
## 1              FLOOD   144022037057
## 2  HURRICANE/TYPHOON    69305840000
## 3        STORM SURGE    43193536000
## 4            TORNADO    24935939545
## 5        FLASH FLOOD    16047794571
## 6               HAIL    15048722103
## 7          HURRICANE    11812819010
## 8     TROPICAL STORM     7653335550
## 9          HIGH WIND     5259785375
## 10          WILDFIRE     4759064000
## 11  STORM SURGE/TIDE     4641188000
## 12         TSTM WIND     4482361440
## 13         ICE STORM     3643555810
## 14 THUNDERSTORM WIND     3399282992
## 15    HURRICANE OPAL     3172846000
\end{verbatim}

\begin{Shaded}
\begin{Highlighting}[]
\NormalTok{crop}
\end{Highlighting}
\end{Shaded}

\begin{verbatim}
##               EVTYPE  cropDamage
## 1            DROUGHT 13922066000
## 2              FLOOD  5422810400
## 3          HURRICANE  2741410000
## 4               HAIL  2614127070
## 5  HURRICANE/TYPHOON  2607872800
## 6        FLASH FLOOD  1343915000
## 7       EXTREME COLD  1292473000
## 8       FROST/FREEZE  1094086000
## 9         HEAVY RAIN   728399800
## 10    TROPICAL STORM   677836000
## 11         HIGH WIND   633561300
## 12         TSTM WIND   553947350
## 13    EXCESSIVE HEAT   492402000
## 14 THUNDERSTORM WIND   414354000
## 15              HEAT   401411500
\end{verbatim}

A pair of graphs of total property damage and total crop damage affected
by these severe weather events is given in the following.

\begin{Shaded}
\begin{Highlighting}[]
\NormalTok{propertyPlot <-}\StringTok{ }\KeywordTok{qplot}\NormalTok{(EVTYPE, }\DataTypeTok{data =}\NormalTok{ property, }\DataTypeTok{weight =}\NormalTok{ propertyDamage, }\DataTypeTok{geom =} \StringTok{"bar"}\NormalTok{, }\DataTypeTok{binwidth =} \OtherTok{NULL}\NormalTok{) }\OperatorTok{+}\StringTok{ }
\StringTok{    }\KeywordTok{theme}\NormalTok{(}\DataTypeTok{axis.text.x =} \KeywordTok{element_text}\NormalTok{(}\DataTypeTok{angle =} \DecValTok{45}\NormalTok{, }\DataTypeTok{hjust =} \DecValTok{1}\NormalTok{)) }\OperatorTok{+}\StringTok{ }\KeywordTok{scale_y_continuous}\NormalTok{(}\StringTok{"Property Damage in US dollars"}\NormalTok{)}\OperatorTok{+}\StringTok{ }
\StringTok{    }\KeywordTok{xlab}\NormalTok{(}\StringTok{"Severe Weather Type"}\NormalTok{) }\OperatorTok{+}\StringTok{ }\KeywordTok{ggtitle}\NormalTok{(}\StringTok{"Total Property Damage by}\CharTok{\textbackslash{}n}\StringTok{ Severe Weather Events in}\CharTok{\textbackslash{}n}\StringTok{ the U.S. from 1995 - 2011"}\NormalTok{)}

\NormalTok{cropPlot<-}\StringTok{ }\KeywordTok{qplot}\NormalTok{(EVTYPE, }\DataTypeTok{data =}\NormalTok{ crop, }\DataTypeTok{weight =}\NormalTok{ cropDamage, }\DataTypeTok{geom =} \StringTok{"bar"}\NormalTok{, }\DataTypeTok{binwidth =} \OtherTok{NULL}\NormalTok{) }\OperatorTok{+}\StringTok{ }
\StringTok{    }\KeywordTok{theme}\NormalTok{(}\DataTypeTok{axis.text.x =} \KeywordTok{element_text}\NormalTok{(}\DataTypeTok{angle =} \DecValTok{45}\NormalTok{, }\DataTypeTok{hjust =} \DecValTok{1}\NormalTok{)) }\OperatorTok{+}\StringTok{ }\KeywordTok{scale_y_continuous}\NormalTok{(}\StringTok{"Crop Damage in US dollars"}\NormalTok{) }\OperatorTok{+}\StringTok{ }
\StringTok{    }\KeywordTok{xlab}\NormalTok{(}\StringTok{"Severe Weather Type"}\NormalTok{) }\OperatorTok{+}\StringTok{ }\KeywordTok{ggtitle}\NormalTok{(}\StringTok{"Total Crop Damage by }\CharTok{\textbackslash{}n}\StringTok{Severe Weather Events in}\CharTok{\textbackslash{}n}\StringTok{ the U.S. from 1995 - 2011"}\NormalTok{)}
\KeywordTok{grid.arrange}\NormalTok{(propertyPlot, cropPlot, }\DataTypeTok{ncol =} \DecValTok{2}\NormalTok{)}
\end{Highlighting}
\end{Shaded}

\includegraphics{storm_data_files/figure-latex/unnamed-chunk-11-1.pdf}

For the goal of finding out the \textbf{flood} and
\textbf{hurricane/typhoon} cause most property damage; \textbf{drought}
and \textbf{flood} causes most crop damage in the United States from
1995 to 2011, two histogram plots are created and displayed
side-by-side.

\subsubsection{Conclusion}\label{conclusion}

The interpretation from this study is described as \textbf{excessive
heat} and \textbf{tornado} are most harmful with respect to population
health, while \textbf{flood}, \textbf{drought}, and
\textbf{hurricane/typhoon} have the greatest economic consequences.


\end{document}
